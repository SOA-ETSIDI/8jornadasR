% Created 2016-11-17 jue 15:11
% Intended LaTeX compiler: pdflatex
\documentclass[spanish, xcolor={usenames,svgnames,dvipsnames}]{beamer}
\usepackage[utf8]{inputenc}
\usepackage[T1]{fontenc}
\usepackage{graphicx}
\usepackage{grffile}
\usepackage{longtable}
\usepackage{wrapfig}
\usepackage{rotating}
\usepackage[normalem]{ulem}
\usepackage{amsmath}
\usepackage{textcomp}
\usepackage{amssymb}
\usepackage{capt-of}
\usepackage{hyperref}
\usepackage{color}
\usepackage{listings}
\usecolortheme{rose}
\setbeamercolor{alerted text}{fg=DarkBlue}
\setbeamerfont{alerted text}{series=\mdseries}
\setbeamerfont{block title}{series=\scshape}
\setbeamercolor{block title}{bg=structure.fg!40!bg!50!bg}
\setbeamercolor{block body}{use=block title,bg=block title.bg!50}
\setbeamertemplate{navigation symbols}{}
\setbeamertemplate{blocks}[rounded][shadow=false]
\AtBeginSection[]{\begin{frame}{Índice}\tableofcontents[currentsection]\end{frame}}
\usepackage[spanish]{babel}
\usepackage{mathpazo}
\hypersetup{colorlinks=true, linkcolor=Blue, urlcolor=Blue}
\logo{\makebox[0.95\paperwidth]{\includegraphics[width=1.5cm,keepaspectratio]{images/LogoUPM.pdf}\hfill\includegraphics[width=1.5cm,keepaspectratio]{images/LogoETSIDI.pdf}}}
\titlegraphic{\includegraphics[width=0.7\textwidth]{images/entrada-escuela}}
\usetheme{default}
\usefonttheme{serif}
\author{Oscar Perpiñán Lamigueiro}
\date{ETSIDI - UPM}
\title{Gestión académica de una escuela universitaria con R y shiny}
\hypersetup{
 pdfauthor={Oscar Perpiñán Lamigueiro},
 pdftitle={Gestión académica de una escuela universitaria con R y shiny},
 pdfkeywords={},
 pdfsubject={},
 pdfcreator={Emacs 24.5.1 (Org mode 9.0)}, 
 pdflang={Spanish}}
\begin{document}

\maketitle

\section{Introducción}
\label{sec:org58ac41f}

\begin{frame}[label={sec:org57d72ee}]{Contexto}
\begin{itemize}
\item ETS de Ingeniería con 2700 alumnos en 10 titulaciones diferentes.
\item Numerosos grupos de matriculación (horarios).
\item Alta ocupación de aulas.
\item Planes Bolonia:
\begin{itemize}
\item Evaluación continua (frecuentes exámenes)
\item Flexibilidad de técnicas docentes
\end{itemize}
\end{itemize}
\end{frame}

\begin{frame}[label={sec:orga5f1ca1}]{Contexto}
\begin{block}{Necesaria renovación de herramientas}
\begin{itemize}
\item Gestión de aulas
\item Elaboración y publicación de horarios docentes y calendario escolar
\item Aprobación, modificación y publicación de horarios del profesorado
\end{itemize}
\end{block}
\end{frame}


\section{Aplicaciones web}
\label{sec:org13e4080}

\begin{frame}[label={sec:org990bb96}]{Catálogo}
\begin{block}{Repositorio}
\url{https://github.com/SOA-ETSIDI/}
\end{block}
\begin{block}{}
\begin{itemize}
\item Horarios del profesorado
\item Horarios docentes
\item Calendario
\item Exámenes (en construcción)
\end{itemize}
\end{block}
\end{frame}


\section{Horarios del profesorado}
\label{sec:orgd7f5726}
\begin{frame}[label={sec:orgaa7e1fe}]{Horarios del profesorado}
\begin{block}{Procedimiento}
\begin{itemize}
\item Cada profesor tiene un enlace único que le permite cumplimentar (y modificar) sus horarios de docencia y tutorías.
\item Cada director del departamento tiene un enlace privado para revisar estos horarios y publicarlos.
\item Una web recopila los horarios de todos los profesores, una vez publicados.
\item Otra web permite la búsqueda libre.
\end{itemize}
\end{block}
\end{frame}

\begin{frame}[label={sec:org8c1140a}]{}
Cada profesor tiene un enlace único que le permite cumplimentar (y modificar) sus horarios de docencia y tutorías: \url{http://programas.etsidi.upm.es/SOA/docencia/?XYZ}

\begin{center}
\includegraphics[width=.9\linewidth]{images/formulario-docencia.png}
\end{center}
\end{frame}

\begin{frame}[label={sec:orgecc2c45}]{}
Una web recopila los horarios de todos los profesores: \url{http://programas.etsidi.upm.es/SOA/tutorias/}

\begin{center}
\includegraphics[width=.9\linewidth]{images/tutorias.png}
\end{center}
\end{frame}

\begin{frame}[label={sec:org23fd66a}]{}
Otra web permite la búsqueda libre: \url{http://programas.etsidi.upm.es/SOA/buscador/}
\begin{center}
\includegraphics[width=.9\linewidth]{images/buscador.png}
\end{center}
\end{frame}



\begin{frame}[label={sec:orgcb0d0e8}]{Horarios del profesorado}
\begin{block}{Aspectos importantes del código}
\begin{itemize}
\item Aplicaciones basadas en shiny con la ayuda de \href{https://daattali.com/shiny/shinyjs-demo/}{shinyjs}
\item \href{https://github.com/Rdatatable/data.table/wiki}{DT} (interfaz para DataTables de JavaScript) en modo lectura y escritura (\href{https://yihui.shinyapps.io/DT-proxy/}{dataTableProxy}).
\item \href{https://github.com/Rdatatable/data.table/wiki}{data.table} detrás del escenario
\item PDF generado automáticamente a través de \href{http://orgmode.org/}{Org} (y \href{https://www.ctan.org/pkg/tcolorbox}{tcolorbox} de \LaTeX{})
\end{itemize}
\end{block}
\end{frame}

\begin{frame}[label={sec:orgca2ad73}]{Ejemplo del PDF}
\begin{center}
\includegraphics[width=.9\linewidth]{images/tutorias_OPL.pdf}
\end{center}
\end{frame}


\section{Horarios docentes}
\label{sec:org6945691}

\begin{frame}[label={sec:org19a667b}]{}
\begin{block}{}
\url{https://github.com/SOA-ETSIDI/horarios}

\begin{center}
\includegraphics[width=.9\linewidth]{images/horarios.png}
\end{center}
\end{block}
\end{frame}

\begin{frame}[label={sec:orgf475314}]{}
\begin{block}{}
\url{https://github.com/SOA-ETSIDI/horarios}

\begin{center}
\includegraphics[width=.9\linewidth]{images/horarios_tt.png}
\end{center}
\end{block}
\end{frame}

\begin{frame}[label={sec:org8875b87}]{Aspectos importantes del código}
\begin{itemize}
\item Aplicación shiny basada en \href{http://jrowen.github.io/rhandsontable/}{rhandsontable} (tabla editable)
\item Produce:
\begin{itemize}
\item Ficheros PDF mediante \href{https://github.com/SOA-ETSIDI/horarios/blob/master/csv2tt.R}{una función} basada en una \href{https://github.com/SOA-ETSIDI/horarios/blob/master/timetable.tex}{plantilla \LaTeX{}} que usa \href{https://en.wikipedia.org/wiki/PGF/TikZ}{TikZ}
\item Ficheros iCalendar mediante dos \href{https://github.com/SOA-ETSIDI/misc/blob/master/funciones.R}{funciones propias}.
\end{itemize}
\end{itemize}
\end{frame}

\begin{frame}[label={sec:org36da7f1}]{Disponibles en servidor}
\url{http://programas.etsidi.upm.es/webdav/horarios/}

\begin{center}
\includegraphics[width=.9\linewidth]{images/D307_1.pdf}
\end{center}
\end{frame}


\section{Calendario escolar}
\label{sec:org5965db1}
\begin{frame}[label={sec:org8f7385e}]{Aspectos importantes del código}
\begin{itemize}
\item Aplicación shiny basada en \href{http://jrowen.github.io/rhandsontable/}{rhandsontable} (tabla editable)
\item Produce:
\begin{itemize}
\item Ficheros PDF mediante \href{https://github.com/SOA-ETSIDI/calendario/blob/master/csv2pdf.R}{una función} basada en una \href{https://github.com/SOA-ETSIDI/calendario/blob/master/documentCal.tex}{plantilla \LaTeX{}} que usa \href{https://en.wikipedia.org/wiki/PGF/TikZ}{TikZ}
\item Ficheros iCalendar mediante dos \href{https://github.com/SOA-ETSIDI/calendario/blob/master/csv2ics.R}{funciones propias}.
\end{itemize}
\end{itemize}
\begin{block}{}
\url{https://github.com/SOA-ETSIDI/calendario}
\end{block}
\end{frame}

\begin{frame}[label={sec:org3c217c5}]{}
\begin{center}
\includegraphics[width=.9\linewidth]{images/webdav_cal.png}
\end{center}
\end{frame}

\begin{frame}[label={sec:org1847a9b}]{}
\begin{center}
\includegraphics[width=.9\linewidth]{images/Calendario_Grado_2016_2017.pdf}
\end{center}
\end{frame}
\end{document}